\documentclass{article}


% if you need to pass options to natbib, use, e.g.:
%     \PassOptionsToPackage{numbers, compress}{natbib}
% before loading neurips_2023


% ready for submission


\usepackage{neurips_2023}



\usepackage[utf8]{inputenc} % allow utf-8 input
\usepackage[T1]{fontenc}    % use 8-bit T1 fonts
\usepackage{hyperref}       % hyperlinks
\usepackage{url}            % simple URL typesetting
\usepackage{booktabs}       % professional-quality tables
\usepackage{amsfonts}       % blackboard math symbols
\usepackage{nicefrac}       % compact symbols for 1/2, etc.
\usepackage{microtype}      % microtypography
\usepackage{xcolor}         % colors


\makeatletter
\newif\if@submission
\@submissionfalse

\makeatother
\title{Formatting Instructions For NeurIPS 2023}
\author {
  HU Gangfeng \\ 2022533025 \\ \texttt{hugf2022@shanghaitech.edu.cn}
  \And
  TENG Zhihao \\ 2022533050 \\ \texttt{tengzhh2022@shanghaitech.edu.cn}
  \And
  QIN Chao \\ 2022533137 \\\texttt{qinchao2022@shanghaitech.edu.cn}
}

\begin{document}


\maketitle


\begin{abstract}
  Improving classification accuracy is a key issue to ad- vancing brain computer interface (BCI) research from lab- oratory to real world applications. 
This article presents a high accuracy EEG signal classification method using sin- gle trial EEG signal to detect left and right finger move- ment. 
We apply an optimal temporal filter to remove  irrelevant signal and subsequently extract key features from spatial patterns of EEG signal to perform classification. 
Specifically, the proposed method transforms the original EEG signal into a spatial pattern and applies the RBFfea- ture selection method to generate robust feature. 
Classification is performed by the SVM and our experimental result shows that the classification accuracy of the proposed method reaches 90\% as compared to the current reported best accuracy of 84\%.
\end{abstract}


\section{Introduction}

A brain-computer interface (BCI) is a communication system that does not depend on the brain's normal output pathways of peripheral nerves and muscles. At present, eletroencephalography(EEG) is one of the most prevailing signals used in non-invasive BCI systems.
There are various kinds of EEG based  BCIs  categorized by the signals used. Typical signals include slow cortical potential, rhythms, EEG (de)synchronization evoked by motor imagery, steady-state visual evoked poten- tial, P300 potential, etc. EEG signals evoked by limb move- ment or motor imagery are of interest to this paper.
The preparation, actual operation and mental imagina- tion of limb movements activate similar EEG changes at sensorimotor areas on the  scalp. When  such regions be- come activated, EEG activities display an amplitude atten- uation or event-related desynchronization (ERD). For in- stance, imagination of right-hand or left-hand movement


results in the most prominent ERD localized over the cor- responding sensorimotor cortex. However, ERD is subject- related,i.e. different subjects have different spatial localiza- tions of ERD. This leads to difficulty when extracting fea- tures for classification.
Pfurtscheller et. al. [6] extracted motor imagery signals from C3 and C4 EEG Channels to build an online BCI sys- tem. The features presented to the classifier were short-term power spectra in pre-define frequency bands. This system using a LVQ algorithm achieved an accuracy of approxi- mately 80% for 3 subjects.
Studies  showed  that  the  position  of  ERD  may  vary from subject to subject, and are not necessarily located be- neath  electrode  positions  C3  and  C4  [5].  As  such,  us- ing more channels of signals may improve performance.
Mller-Gerking et. al. [4] proposed to use Common Spa-
tial Patterns (CSP) for the classification of motor execution or imagery  signals. The CSP method resulted in  signifi- cant improvement to performances  as compared to their previous work in  [6].
In this paper, we combined CSP and Principal Compo- nent Analysis (PCA) to improve the CSP feature classifi- cation. The resulted transformation is equivalent to a set of spatial filters optimized to distinguish between the left and right hand movement or motor imagery. In addition, temporal filtering was applied to reduce noise. In the past, the selection of frequency bands was limited to a few pre- defined bands [4, 5]. In this paper, we investigated the ef- fects of temporal  filtering for  specific  subject by  an ex- haustive search over all the frequency bands. We showed that classification performance could be improved signifi- cantly by applying proper band-pass filter. To further en- hance recognition accuracy, a Radial Basis Function (RBF) based feature selection and generation algorithm  [3] was adapted. We applied the Orthogonal Least Square (OLS) algorithm [3] to feature selection and generation. Using a Support Vector Machine (SVM) classifier on the features found, we achieved 90% accuracy on a self-paced finger- taping dataset, the current best result in the literature on this

dataset.
The organization of the paper is as follows. Section 2 in- troduces the feature extraction by the combination of CSP and PCA. Section 3 presents the feature selection and gen- eration algorithms. Section 4 discusses the effects of differ- ent parameters on the recognition performance and present comparative experiment results. Finally, we conclude our paper and discuss some future work.


\section{}






\end{document}